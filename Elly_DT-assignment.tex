% Options for packages loaded elsewhere
\PassOptionsToPackage{unicode}{hyperref}
\PassOptionsToPackage{hyphens}{url}
%
\documentclass[
]{article}
\usepackage{amsmath,amssymb}
\usepackage{iftex}
\ifPDFTeX
  \usepackage[T1]{fontenc}
  \usepackage[utf8]{inputenc}
  \usepackage{textcomp} % provide euro and other symbols
\else % if luatex or xetex
  \usepackage{unicode-math} % this also loads fontspec
  \defaultfontfeatures{Scale=MatchLowercase}
  \defaultfontfeatures[\rmfamily]{Ligatures=TeX,Scale=1}
\fi
\usepackage{lmodern}
\ifPDFTeX\else
  % xetex/luatex font selection
\fi
% Use upquote if available, for straight quotes in verbatim environments
\IfFileExists{upquote.sty}{\usepackage{upquote}}{}
\IfFileExists{microtype.sty}{% use microtype if available
  \usepackage[]{microtype}
  \UseMicrotypeSet[protrusion]{basicmath} % disable protrusion for tt fonts
}{}
\makeatletter
\@ifundefined{KOMAClassName}{% if non-KOMA class
  \IfFileExists{parskip.sty}{%
    \usepackage{parskip}
  }{% else
    \setlength{\parindent}{0pt}
    \setlength{\parskip}{6pt plus 2pt minus 1pt}}
}{% if KOMA class
  \KOMAoptions{parskip=half}}
\makeatother
\usepackage{xcolor}
\usepackage[margin=1in]{geometry}
\usepackage{color}
\usepackage{fancyvrb}
\newcommand{\VerbBar}{|}
\newcommand{\VERB}{\Verb[commandchars=\\\{\}]}
\DefineVerbatimEnvironment{Highlighting}{Verbatim}{commandchars=\\\{\}}
% Add ',fontsize=\small' for more characters per line
\usepackage{framed}
\definecolor{shadecolor}{RGB}{248,248,248}
\newenvironment{Shaded}{\begin{snugshade}}{\end{snugshade}}
\newcommand{\AlertTok}[1]{\textcolor[rgb]{0.94,0.16,0.16}{#1}}
\newcommand{\AnnotationTok}[1]{\textcolor[rgb]{0.56,0.35,0.01}{\textbf{\textit{#1}}}}
\newcommand{\AttributeTok}[1]{\textcolor[rgb]{0.13,0.29,0.53}{#1}}
\newcommand{\BaseNTok}[1]{\textcolor[rgb]{0.00,0.00,0.81}{#1}}
\newcommand{\BuiltInTok}[1]{#1}
\newcommand{\CharTok}[1]{\textcolor[rgb]{0.31,0.60,0.02}{#1}}
\newcommand{\CommentTok}[1]{\textcolor[rgb]{0.56,0.35,0.01}{\textit{#1}}}
\newcommand{\CommentVarTok}[1]{\textcolor[rgb]{0.56,0.35,0.01}{\textbf{\textit{#1}}}}
\newcommand{\ConstantTok}[1]{\textcolor[rgb]{0.56,0.35,0.01}{#1}}
\newcommand{\ControlFlowTok}[1]{\textcolor[rgb]{0.13,0.29,0.53}{\textbf{#1}}}
\newcommand{\DataTypeTok}[1]{\textcolor[rgb]{0.13,0.29,0.53}{#1}}
\newcommand{\DecValTok}[1]{\textcolor[rgb]{0.00,0.00,0.81}{#1}}
\newcommand{\DocumentationTok}[1]{\textcolor[rgb]{0.56,0.35,0.01}{\textbf{\textit{#1}}}}
\newcommand{\ErrorTok}[1]{\textcolor[rgb]{0.64,0.00,0.00}{\textbf{#1}}}
\newcommand{\ExtensionTok}[1]{#1}
\newcommand{\FloatTok}[1]{\textcolor[rgb]{0.00,0.00,0.81}{#1}}
\newcommand{\FunctionTok}[1]{\textcolor[rgb]{0.13,0.29,0.53}{\textbf{#1}}}
\newcommand{\ImportTok}[1]{#1}
\newcommand{\InformationTok}[1]{\textcolor[rgb]{0.56,0.35,0.01}{\textbf{\textit{#1}}}}
\newcommand{\KeywordTok}[1]{\textcolor[rgb]{0.13,0.29,0.53}{\textbf{#1}}}
\newcommand{\NormalTok}[1]{#1}
\newcommand{\OperatorTok}[1]{\textcolor[rgb]{0.81,0.36,0.00}{\textbf{#1}}}
\newcommand{\OtherTok}[1]{\textcolor[rgb]{0.56,0.35,0.01}{#1}}
\newcommand{\PreprocessorTok}[1]{\textcolor[rgb]{0.56,0.35,0.01}{\textit{#1}}}
\newcommand{\RegionMarkerTok}[1]{#1}
\newcommand{\SpecialCharTok}[1]{\textcolor[rgb]{0.81,0.36,0.00}{\textbf{#1}}}
\newcommand{\SpecialStringTok}[1]{\textcolor[rgb]{0.31,0.60,0.02}{#1}}
\newcommand{\StringTok}[1]{\textcolor[rgb]{0.31,0.60,0.02}{#1}}
\newcommand{\VariableTok}[1]{\textcolor[rgb]{0.00,0.00,0.00}{#1}}
\newcommand{\VerbatimStringTok}[1]{\textcolor[rgb]{0.31,0.60,0.02}{#1}}
\newcommand{\WarningTok}[1]{\textcolor[rgb]{0.56,0.35,0.01}{\textbf{\textit{#1}}}}
\usepackage{graphicx}
\makeatletter
\def\maxwidth{\ifdim\Gin@nat@width>\linewidth\linewidth\else\Gin@nat@width\fi}
\def\maxheight{\ifdim\Gin@nat@height>\textheight\textheight\else\Gin@nat@height\fi}
\makeatother
% Scale images if necessary, so that they will not overflow the page
% margins by default, and it is still possible to overwrite the defaults
% using explicit options in \includegraphics[width, height, ...]{}
\setkeys{Gin}{width=\maxwidth,height=\maxheight,keepaspectratio}
% Set default figure placement to htbp
\makeatletter
\def\fps@figure{htbp}
\makeatother
\setlength{\emergencystretch}{3em} % prevent overfull lines
\providecommand{\tightlist}{%
  \setlength{\itemsep}{0pt}\setlength{\parskip}{0pt}}
\setcounter{secnumdepth}{-\maxdimen} % remove section numbering
\ifLuaTeX
  \usepackage{selnolig}  % disable illegal ligatures
\fi
\usepackage{bookmark}
\IfFileExists{xurl.sty}{\usepackage{xurl}}{} % add URL line breaks if available
\urlstyle{same}
\hypersetup{
  pdftitle={Data Typesin R},
  pdfauthor={Elly Ochieng'},
  hidelinks,
  pdfcreator={LaTeX via pandoc}}

\title{Data Typesin R}
\author{Elly Ochieng'}
\date{2024-10-30}

\begin{document}
\maketitle

\subsection{R Markdown}\label{r-markdown}

\begin{Shaded}
\begin{Highlighting}[]
\CommentTok{\# Data Types in R}

\CommentTok{\# R treats everything as an object, and the simplest data objects are known as atomic data types.}

\CommentTok{\# Atomic data types allow the creation of atomic vectors.}
\CommentTok{\# Common atomic data types include:}
\CommentTok{\# {-} Numeric (integer and double)}
\CommentTok{\# {-} Character}
\CommentTok{\# {-} Logical}
\CommentTok{\# {-} Complex}
\CommentTok{\# {-} Raw}

\CommentTok{\# Checking if an object is atomic}
\FunctionTok{is.atomic}\NormalTok{(}\DecValTok{3}\NormalTok{)            }\CommentTok{\# TRUE, 3 is a numeric atomic vector}
\end{Highlighting}
\end{Shaded}

\begin{verbatim}
## [1] TRUE
\end{verbatim}

\begin{Shaded}
\begin{Highlighting}[]
\FunctionTok{is.atomic}\NormalTok{(}\StringTok{"R CODER"}\NormalTok{)    }\CommentTok{\# TRUE, "R CODER" is a character atomic vector}
\end{Highlighting}
\end{Shaded}

\begin{verbatim}
## [1] TRUE
\end{verbatim}

\begin{Shaded}
\begin{Highlighting}[]
\CommentTok{\# Functions to check data type in R}
\FunctionTok{typeof}\NormalTok{(}\DecValTok{1}\NormalTok{)               }\CommentTok{\# "double", shows the internal type}
\end{Highlighting}
\end{Shaded}

\begin{verbatim}
## [1] "double"
\end{verbatim}

\begin{Shaded}
\begin{Highlighting}[]
\FunctionTok{class}\NormalTok{(}\DecValTok{2}\NormalTok{)                }\CommentTok{\# "numeric", shows the object\textquotesingle{}s class}
\end{Highlighting}
\end{Shaded}

\begin{verbatim}
## [1] "numeric"
\end{verbatim}

\begin{Shaded}
\begin{Highlighting}[]
\FunctionTok{storage.mode}\NormalTok{(}\DecValTok{3}\NormalTok{)        }\CommentTok{\# "double", shows storage mode}
\end{Highlighting}
\end{Shaded}

\begin{verbatim}
## [1] "double"
\end{verbatim}

\begin{Shaded}
\begin{Highlighting}[]
\FunctionTok{mode}\NormalTok{(}\DecValTok{4}\NormalTok{)                 }\CommentTok{\# "numeric", another way to check type}
\end{Highlighting}
\end{Shaded}

\begin{verbatim}
## [1] "numeric"
\end{verbatim}

\begin{Shaded}
\begin{Highlighting}[]
\FunctionTok{str}\NormalTok{(}\DecValTok{5}\NormalTok{)                  }\CommentTok{\# Displays the structure of the object, shows it is numeric}
\end{Highlighting}
\end{Shaded}

\begin{verbatim}
##  num 5
\end{verbatim}

\begin{Shaded}
\begin{Highlighting}[]
\CommentTok{\# Example of changing an object\textquotesingle{}s class}
\NormalTok{x }\OtherTok{\textless{}{-}} \DecValTok{1}                  \CommentTok{\# Assigning numeric value 1}
\FunctionTok{class}\NormalTok{(x)                }\CommentTok{\# "numeric", initial class}
\end{Highlighting}
\end{Shaded}

\begin{verbatim}
## [1] "numeric"
\end{verbatim}

\begin{Shaded}
\begin{Highlighting}[]
\FunctionTok{class}\NormalTok{(x) }\OtherTok{\textless{}{-}} \StringTok{"My\_class"}  \CommentTok{\# Changing class to "My\_class"}
\FunctionTok{class}\NormalTok{(x)                }\CommentTok{\# "My\_class", confirms class change}
\end{Highlighting}
\end{Shaded}

\begin{verbatim}
## [1] "My_class"
\end{verbatim}

\begin{Shaded}
\begin{Highlighting}[]
\FunctionTok{typeof}\NormalTok{(x)               }\CommentTok{\# "double", still shows it\textquotesingle{}s a double}
\end{Highlighting}
\end{Shaded}

\begin{verbatim}
## [1] "double"
\end{verbatim}

\begin{Shaded}
\begin{Highlighting}[]
\CommentTok{\# Summary of Type Functions}
\CommentTok{\# Outputs of typeof, storage.mode, and mode for various types}
\CommentTok{\# | Function       | logical | integer | double | character | raw  |}
\CommentTok{\# |{-}{-}{-}{-}{-}{-}{-}{-}{-}{-}{-}{-}{-}{-}{-}{-}|{-}{-}{-}{-}{-}{-}{-}{-}{-}|{-}{-}{-}{-}{-}{-}{-}{-}{-}|{-}{-}{-}{-}{-}{-}{-}{-}|{-}{-}{-}{-}{-}{-}{-}{-}{-}{-}{-}|{-}{-}{-}{-}{-}{-}|}
\CommentTok{\# | typeof         | logical | numeric | double | character | raw  |}
\CommentTok{\# | storage.mode   | logical | numeric | double | character | raw  |}
\CommentTok{\# | mode           | logical | integer | double | character | raw  |}

\CommentTok{\# Numeric Data Types}
\CommentTok{\# Numeric types consist of double and integer.}
\FunctionTok{mode}\NormalTok{(}\DecValTok{55}\NormalTok{)                }\CommentTok{\# "numeric"}
\end{Highlighting}
\end{Shaded}

\begin{verbatim}
## [1] "numeric"
\end{verbatim}

\begin{Shaded}
\begin{Highlighting}[]
\FunctionTok{is.numeric}\NormalTok{(}\DecValTok{3}\NormalTok{)          }\CommentTok{\# TRUE, confirms it\textquotesingle{}s numeric}
\end{Highlighting}
\end{Shaded}

\begin{verbatim}
## [1] TRUE
\end{verbatim}

\begin{Shaded}
\begin{Highlighting}[]
\CommentTok{\# Double or Real Data Type}
\CommentTok{\# Double{-}precision representation is default for all numbers}
\FunctionTok{typeof}\NormalTok{(}\DecValTok{2}\NormalTok{)              }\CommentTok{\# "double"}
\end{Highlighting}
\end{Shaded}

\begin{verbatim}
## [1] "double"
\end{verbatim}

\begin{Shaded}
\begin{Highlighting}[]
\FunctionTok{typeof}\NormalTok{(}\ConstantTok{Inf}\NormalTok{)            }\CommentTok{\# "double", shows infinity}
\end{Highlighting}
\end{Shaded}

\begin{verbatim}
## [1] "double"
\end{verbatim}

\begin{Shaded}
\begin{Highlighting}[]
\FunctionTok{typeof}\NormalTok{(}\SpecialCharTok{{-}}\ConstantTok{Inf}\NormalTok{)           }\CommentTok{\# "double", shows negative infinity}
\end{Highlighting}
\end{Shaded}

\begin{verbatim}
## [1] "double"
\end{verbatim}

\begin{Shaded}
\begin{Highlighting}[]
\FunctionTok{typeof}\NormalTok{(}\ConstantTok{NaN}\NormalTok{)            }\CommentTok{\# "double", represents "Not a Number"}
\end{Highlighting}
\end{Shaded}

\begin{verbatim}
## [1] "double"
\end{verbatim}

\begin{Shaded}
\begin{Highlighting}[]
\FunctionTok{typeof}\NormalTok{(}\FloatTok{3.12e3}\NormalTok{)         }\CommentTok{\# "double", shows scientific notation}
\end{Highlighting}
\end{Shaded}

\begin{verbatim}
## [1] "double"
\end{verbatim}

\begin{Shaded}
\begin{Highlighting}[]
\FunctionTok{typeof}\NormalTok{(}\DecValTok{0xbade}\NormalTok{)         }\CommentTok{\# "double", hexadecimal notation}
\end{Highlighting}
\end{Shaded}

\begin{verbatim}
## [1] "double"
\end{verbatim}

\begin{Shaded}
\begin{Highlighting}[]
\CommentTok{\# Check if an object is double}
\FunctionTok{is.double}\NormalTok{(}\DecValTok{2}\NormalTok{)          }\CommentTok{\# TRUE}
\end{Highlighting}
\end{Shaded}

\begin{verbatim}
## [1] TRUE
\end{verbatim}

\begin{Shaded}
\begin{Highlighting}[]
\FunctionTok{is.double}\NormalTok{(}\FloatTok{2.8}\NormalTok{)        }\CommentTok{\# TRUE}
\end{Highlighting}
\end{Shaded}

\begin{verbatim}
## [1] TRUE
\end{verbatim}

\begin{Shaded}
\begin{Highlighting}[]
\CommentTok{\# Integer Data Type}
\CommentTok{\# Create integers by appending L to a number}
\NormalTok{y }\OtherTok{\textless{}{-}} \DecValTok{2}\DataTypeTok{L}                \CommentTok{\# Creates integer}
\FunctionTok{typeof}\NormalTok{(y)              }\CommentTok{\# "integer"}
\end{Highlighting}
\end{Shaded}

\begin{verbatim}
## [1] "integer"
\end{verbatim}

\begin{Shaded}
\begin{Highlighting}[]
\FunctionTok{is.integer}\NormalTok{(}\DecValTok{3}\NormalTok{)          }\CommentTok{\# FALSE, 3 is a double}
\end{Highlighting}
\end{Shaded}

\begin{verbatim}
## [1] FALSE
\end{verbatim}

\begin{Shaded}
\begin{Highlighting}[]
\FunctionTok{is.integer}\NormalTok{(}\DecValTok{3}\DataTypeTok{L}\NormalTok{)         }\CommentTok{\# TRUE, 3L is an integer}
\end{Highlighting}
\end{Shaded}

\begin{verbatim}
## [1] TRUE
\end{verbatim}

\begin{Shaded}
\begin{Highlighting}[]
\CommentTok{\# Logical Data Type}
\CommentTok{\# Composed of TRUE, FALSE, and NA}
\NormalTok{t }\OtherTok{\textless{}{-}} \ConstantTok{TRUE}              \CommentTok{\# Assign TRUE}
\NormalTok{f }\OtherTok{\textless{}{-}} \ConstantTok{FALSE}             \CommentTok{\# Assign FALSE}
\NormalTok{n }\OtherTok{\textless{}{-}} \ConstantTok{NA}                \CommentTok{\# Assign NA}
\FunctionTok{typeof}\NormalTok{(t)              }\CommentTok{\# "logical"}
\end{Highlighting}
\end{Shaded}

\begin{verbatim}
## [1] "logical"
\end{verbatim}

\begin{Shaded}
\begin{Highlighting}[]
\FunctionTok{typeof}\NormalTok{(f)              }\CommentTok{\# "logical"}
\end{Highlighting}
\end{Shaded}

\begin{verbatim}
## [1] "logical"
\end{verbatim}

\begin{Shaded}
\begin{Highlighting}[]
\FunctionTok{typeof}\NormalTok{(n)              }\CommentTok{\# "logical"}
\end{Highlighting}
\end{Shaded}

\begin{verbatim}
## [1] "logical"
\end{verbatim}

\begin{Shaded}
\begin{Highlighting}[]
\CommentTok{\# Check if an object is logical}
\FunctionTok{is.logical}\NormalTok{(T)          }\CommentTok{\# TRUE}
\end{Highlighting}
\end{Shaded}

\begin{verbatim}
## [1] TRUE
\end{verbatim}

\begin{Shaded}
\begin{Highlighting}[]
\FunctionTok{is.logical}\NormalTok{(}\ConstantTok{TRUE}\NormalTok{)       }\CommentTok{\# TRUE}
\end{Highlighting}
\end{Shaded}

\begin{verbatim}
## [1] TRUE
\end{verbatim}

\begin{Shaded}
\begin{Highlighting}[]
\CommentTok{\# Caution with T and F}
\CommentTok{\# Using T and F can override the values}
\NormalTok{F                     }\CommentTok{\# FALSE, the default}
\end{Highlighting}
\end{Shaded}

\begin{verbatim}
## [1] FALSE
\end{verbatim}

\begin{Shaded}
\begin{Highlighting}[]
\NormalTok{a }\OtherTok{\textless{}{-}}\NormalTok{ T                }\CommentTok{\# Assigns TRUE to variable a}
\NormalTok{F }\OtherTok{\textless{}{-}}\NormalTok{ a                }\CommentTok{\# Now F is TRUE}

\CommentTok{\# Complex Data Type}
\CommentTok{\# Includes imaginary numbers}
\DecValTok{1} \SpecialCharTok{+} \DecValTok{3}\DataTypeTok{i}                \CommentTok{\# Represents a complex number}
\end{Highlighting}
\end{Shaded}

\begin{verbatim}
## [1] 1+3i
\end{verbatim}

\begin{Shaded}
\begin{Highlighting}[]
\FunctionTok{typeof}\NormalTok{(}\DecValTok{1} \SpecialCharTok{+} \DecValTok{3}\DataTypeTok{i}\NormalTok{)        }\CommentTok{\# "complex"}
\end{Highlighting}
\end{Shaded}

\begin{verbatim}
## [1] "complex"
\end{verbatim}

\begin{Shaded}
\begin{Highlighting}[]
\FunctionTok{is.complex}\NormalTok{(}\DecValTok{1} \SpecialCharTok{+} \DecValTok{3}\DataTypeTok{i}\NormalTok{)    }\CommentTok{\# TRUE}
\end{Highlighting}
\end{Shaded}

\begin{verbatim}
## [1] TRUE
\end{verbatim}

\begin{Shaded}
\begin{Highlighting}[]
\CommentTok{\# String or Character Data Type}
\CommentTok{\# Character strings are enclosed in quotes}
\NormalTok{character }\OtherTok{\textless{}{-}} \StringTok{"a"}      \CommentTok{\# Assigns a string}
\FunctionTok{typeof}\NormalTok{(character)     }\CommentTok{\# "character"}
\end{Highlighting}
\end{Shaded}

\begin{verbatim}
## [1] "character"
\end{verbatim}

\begin{Shaded}
\begin{Highlighting}[]
\FunctionTok{is.character}\NormalTok{(character) }\CommentTok{\# TRUE}
\end{Highlighting}
\end{Shaded}

\begin{verbatim}
## [1] TRUE
\end{verbatim}

\begin{Shaded}
\begin{Highlighting}[]
\FunctionTok{typeof}\NormalTok{(}\StringTok{\textquotesingle{}R CODER\textquotesingle{}}\NormalTok{)     }\CommentTok{\# "character"}
\end{Highlighting}
\end{Shaded}

\begin{verbatim}
## [1] "character"
\end{verbatim}

\begin{Shaded}
\begin{Highlighting}[]
\FunctionTok{typeof}\NormalTok{(}\StringTok{"R CODER"}\NormalTok{)     }\CommentTok{\# "character"}
\end{Highlighting}
\end{Shaded}

\begin{verbatim}
## [1] "character"
\end{verbatim}

\begin{Shaded}
\begin{Highlighting}[]
\FunctionTok{nchar}\NormalTok{(}\StringTok{"A string"}\NormalTok{)     }\CommentTok{\# 8, counts characters including spaces}
\end{Highlighting}
\end{Shaded}

\begin{verbatim}
## [1] 8
\end{verbatim}

\begin{Shaded}
\begin{Highlighting}[]
\CommentTok{\# Raw Data Type in R}
\CommentTok{\# Holds raw bytes and is less common}
\NormalTok{a }\OtherTok{\textless{}{-}} \FunctionTok{charToRaw}\NormalTok{(}\StringTok{"R CODER"}\NormalTok{)  }\CommentTok{\# Converts string to raw bytes                         \# Outputs raw byte representation}
\FunctionTok{typeof}\NormalTok{(a)                 }\CommentTok{\# "raw"}
\end{Highlighting}
\end{Shaded}

\begin{verbatim}
## [1] "raw"
\end{verbatim}

\begin{Shaded}
\begin{Highlighting}[]
\NormalTok{b }\OtherTok{\textless{}{-}} \FunctionTok{intToBits}\NormalTok{(}\DecValTok{3}\DataTypeTok{L}\NormalTok{)        }\CommentTok{\# Converts integer to raw bits}
\FunctionTok{typeof}\NormalTok{(b)                 }\CommentTok{\# "raw"}
\end{Highlighting}
\end{Shaded}

\begin{verbatim}
## [1] "raw"
\end{verbatim}

\begin{Shaded}
\begin{Highlighting}[]
\FunctionTok{is.raw}\NormalTok{(b)                }\CommentTok{\# TRUE}
\end{Highlighting}
\end{Shaded}

\begin{verbatim}
## [1] TRUE
\end{verbatim}

\begin{Shaded}
\begin{Highlighting}[]
\CommentTok{\# Date and Time Data Type in R}
\CommentTok{\# Dates can be represented using as.Date()}
\NormalTok{date\_example }\OtherTok{\textless{}{-}} \FunctionTok{as.Date}\NormalTok{(}\StringTok{"2024{-}10{-}30"}\NormalTok{)  }\CommentTok{\# Example of creating a date object}

\CommentTok{\# Data Types Coercion in R}
\CommentTok{\# Coerce data types using functions that start with as.}
\CommentTok{\# | Function        | Coerced Data Type |}
\CommentTok{\# |{-}{-}{-}{-}{-}{-}{-}{-}{-}{-}{-}{-}{-}{-}{-}{-}{-}|{-}{-}{-}{-}{-}{-}{-}{-}{-}{-}{-}{-}{-}{-}{-}{-}{-}{-}{-}{-}|}
\CommentTok{\# | as.numeric      | Numeric            |}
\CommentTok{\# | as.integer      | Integer            |}
\CommentTok{\# | as.double       | Double             |}
\CommentTok{\# | as.character    | Character          |}
\CommentTok{\# | as.logical      | Boolean            |}
\CommentTok{\# | as.raw          | Raw                |}

\CommentTok{\# Example of coercing a double to integer}
\NormalTok{a }\OtherTok{\textless{}{-}} \DecValTok{3}                  \CommentTok{\# double by default}
\FunctionTok{typeof}\NormalTok{(a)               }\CommentTok{\# "double"}
\end{Highlighting}
\end{Shaded}

\begin{verbatim}
## [1] "double"
\end{verbatim}

\begin{Shaded}
\begin{Highlighting}[]
\NormalTok{a }\OtherTok{\textless{}{-}} \FunctionTok{as.integer}\NormalTok{(a)      }\CommentTok{\# coerces to integer}
\FunctionTok{typeof}\NormalTok{(a)               }\CommentTok{\# "integer"}
\end{Highlighting}
\end{Shaded}

\begin{verbatim}
## [1] "integer"
\end{verbatim}

\begin{Shaded}
\begin{Highlighting}[]
\CommentTok{\# Coerce logical values}
\NormalTok{b }\OtherTok{\textless{}{-}} \ConstantTok{TRUE}
\NormalTok{b }\OtherTok{\textless{}{-}} \FunctionTok{as.numeric}\NormalTok{(b)      }\CommentTok{\# Coerces TRUE to 1}
\NormalTok{b                       }\CommentTok{\# Outputs 1}
\end{Highlighting}
\end{Shaded}

\begin{verbatim}
## [1] 1
\end{verbatim}

\begin{Shaded}
\begin{Highlighting}[]
\NormalTok{c }\OtherTok{\textless{}{-}} \ConstantTok{FALSE}
\NormalTok{c }\OtherTok{\textless{}{-}} \FunctionTok{as.numeric}\NormalTok{(c)      }\CommentTok{\# Coerces FALSE to 0}
\NormalTok{c                       }\CommentTok{\# Outputs 0}
\end{Highlighting}
\end{Shaded}

\begin{verbatim}
## [1] 0
\end{verbatim}

\begin{Shaded}
\begin{Highlighting}[]
\NormalTok{d }\OtherTok{\textless{}{-}} \ConstantTok{TRUE}
\NormalTok{d }\OtherTok{\textless{}{-}} \FunctionTok{as.character}\NormalTok{(d)    }\CommentTok{\# Coerces TRUE to string "TRUE"}
\NormalTok{d                       }\CommentTok{\# Outputs "TRUE"}
\end{Highlighting}
\end{Shaded}

\begin{verbatim}
## [1] "TRUE"
\end{verbatim}

\begin{Shaded}
\begin{Highlighting}[]
\CommentTok{\# Attempting to coerce incompatible types will yield an error}
\FunctionTok{as.double}\NormalTok{(}\StringTok{"R CODER"}\NormalTok{)    }\CommentTok{\# Outputs NA with a warning: NAs introduced by coercion}
\end{Highlighting}
\end{Shaded}

\begin{verbatim}
## Warning: NAs introduced by coercion
\end{verbatim}

\begin{verbatim}
## [1] NA
\end{verbatim}

\end{document}
